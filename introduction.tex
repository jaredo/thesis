\chapter{Introduction}
Despite the unabated growth in popularity of next-generation sequencing data, genotype microarrays continue to play a significant role in genetic research. Novel variants that contribute to disease risk and other phenotypes continue to be found in genome-wide association studies (GWAS)~\citep{ripke2013genome} and associated meta-analyses~\citep{evangelou2013meta}.  Studies that estimate the additive heritability of phenotypes using microarray data have also furthered our understanding of disease aetiology~\citep{lee2011,lee2012estimating}. Cohorts assayed on microarrays continue to provide interesting results in population genetics~\citep{ralph2012geography,lawson2012inference}.  Hence high quality methods and associated software for the analysis of microarray data are an important component of any bioinformatics toolkit. In this thesis we investigate two important areas of primary analysis for microarray data; genotype calling and haplotype inference. 

Genotype calling  involves converting the bivariate allelic intensities produced by microarrays into hard genotype calls (or the relevant posterior probabilities)  via cluster analysis.  This is obviously the first step in any analysis involving such data.  Haplotype inference (often referred to as phasing) involves deconvolving the two parental haplotypes from this genotype data. Phasing is a crucial component of genotype imputation from dense reference panels which is a standard part of GWAS pipelines. Haplotypes also have utility in their own right, for example in the investigation parent-of-origin effects and for population genetics research.  Statistical methods for both of these problems need to be as accurate as possible whilst remaining computationally tractable.

In chapter two we introduce a model  that can call genotypes for individuals who have been assayed on both microarrays as well being sequenced, the method effectively fuses both pieces of information for more accurate genotypes and less missing data.  The method is also competitive with other state of the art microarray callers when no sequence data is available.  We describe the model and evaluate it and other genotype calling software on 1000 Genomes individuals who have been assayed on the Illumina Omni2.5S microarray.

In chapter three we give an overview of phasing methodology for dealing with cohorts that have differing levels of relatedness. We cover popular techniques for unrelated individuals,  explicit pedigrees and finally the intermediate case of long range phasing in cohorts sampled from population isolates.  We also introduce a simple method for combining unrelated phasing methods with extended pedigree information. This gives a sense of what tools are appropriate for cohorts with various demographies.

In chapter four we evaluate these phasing methods on simulated and real data.  We predominantly focus on data from population isolates in Europe and cohorts with a mixture of pedigrees and nominally unrelated individuals.  To date, there has been limited evaluation of haplotype inference in these scenarios.  We demonstrate, somewhat surprisingly, that the SHAPEIT2 method is incredibly robust across the full spectrum of relatedness.  Combined with our new method for extended pedigrees, we provide evidence that it can even outperform traditional pedigree phasing techniques that take into account explicit pedigree structure.

In chapter five we investigate how SHAPEIT2 scales with sample size.  We introduce some modifications to the SHAPEIT2 algorithm to enable linear scaling for sample sizes up to (at least) 50,000 individuals.  We evaluate the accuracy of this modification on simulated and real data. This new approach compares well with the fastest phasing method that is currently public available, HAPI-UR.


